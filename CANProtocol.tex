[BRIEF DESCRIPTION OF CAN AND WHY ITS BEING USED]\\
SEE IF STUFF IS WORKING
[PRETTY SCHEMATIC/BLOCK DIAGRAM TO DO WITH CAN]\\

[EXPLANATION OF PRETTY PICTURE AND VERY BRIEF SUMMARY]\\

\begin{tikzpicture} [every node/.style={text height=0.7em}]
\matrix(canframe)[matrix of nodes, nodes in empty cells, align=center, style={text width=0.7em}]
{
%	  &   &   &   &   &   &   &   &   &   &   & \rotatebox[]{90}{Request Remote Frame}&
%	  \rotatebox[]{90}{ID Ext.}&
%	  \rotatebox[]{90}{Reserved}&   &   &   &   &   &   &   &   &  lol\\
	0 &
	0 & 0 & 0 & 0 & 0 & 0 & 1 & 0 & 1 & 1 & 1 & 0 &
	0 & 0 & 0 & 0 & 1 & 0 & A & A & A & A & B & B & B & B & 1 & 0 & 1 & 0\\
};
% Matrix is called 'mat' first number is row, second numer is column
\node[black,above,style={text height=3.5em}, text width=0.7em] (SoF) at (canframe-1-1.north){\rotatebox[]{90}{Start Of Frame} };
\draw[thick,black] (canframe-1-1.south west) -| (SoF.north west);
\draw[thick,black,name path=SoFr] (canframe-1-2.south west) -| (SoF.north east);

\node[black,above,style={text height=3.5em, text width=0.7em}] (RFR) at (canframe-1-13.north){\rotatebox[]{90}{Requ. Remote} };
\draw[thick,black, name path=Requ] (canframe-1-13.south west) -| (RFR.north west);
\draw[thick,black, name path=RequR] (canframe-1-14.south west) -| (RFR.north east);



\node[black,above,style={text height=3.5em, text width=0.7em}] (ID) at (canframe-1-14.north){\rotatebox[]{90}{ID Ext.} };
%\draw[thick,black] (canframe-1-13.south west) -| (ID.north west);
%\draw[thick,black] (canframe-1-14.south west) -| (ID.north east);
\node[black,above,style={text height=2.5em, text width=0.7em}] (Resv) at (canframe-1-15.north){\rotatebox[]{90}{Reserved} };
\draw[thick,black, name path=Resvln] (canframe-1-15.south east) |- (Resv.north east);
%\draw[thick,black] (canframe-1-15.north east) |- (canframe-2-15.south east);
%\draw[thick,black] (canframe-2-15.south east) |- (canframe-2-15.south east);
\path[name path=tophoriz] (RFR.north east) -- ++(20em, 0);
\path[name path=CtlDatGuide] (canframe-1-19.north east) -- ++(0,8.7em);
\draw[thick,black,name path=CtlDat, name intersections={of=CtlDatGuide and tophoriz,by={Top}}] (canframe-1-19.south east) -- (Top);

%Node ID
\node [above= 6em of {$(canframe-1-2.west)!0.5!(canframe-1-12.east)$}, align=center] (NodeID) {Node ID};

\path[name path=horiz1](NodeID.west) -- ++(-6em,0);
\draw[->,thick,black,name intersections={of=SoFr and horiz1,by={Int1}}] (NodeID.west) -- (Int1);

\path[name path=horiz2](NodeID.east) -- ++(6em,0);
\draw[->,thick,black,name intersections={of=Requ and horiz2,by={Int2}}] (NodeID.east) -- (Int2);


%11 Bits
\node [above= 3em of {$(canframe-1-2.west)!0.5!(canframe-1-12.east)$}, align=center] (11bits) {11 Bits};

\path[name path=horiz3](11bits.west) -- ++(-6em,0);
\draw[->,thick,black,name intersections={of=SoFr and horiz3,by={Int3}}] (11bits.west) -- (Int3);

\path[name path=horiz4](11bits.east) -- ++(6em,0);
\draw[->,thick,black,name intersections={of=Requ and horiz4,by={Int4}}] (11bits.east) -- (Int4);


%Control
\node [above= 6em of {$(canframe-1-14.west)!0.5!(canframe-1-19.east)$}, align=center] (Control) {Control};

\path[name path=horiz5](Control.west) -- ++(-5em,0);
\draw[->,thick,black,name intersections={of=RequR and horiz5,by={Int5}}] (Control.west) -- (Int5);

\path[name path=horiz6](Control.east) -- ++(5em,0);
\draw[->,thick,black,name intersections={of=CtlDat and horiz6,by={Int6}}] (Control.east) -- (Int6);


%Data Length
\node [above = 3em of {$(canframe-1-16.west)!0.5!(canframe-1-19.east)$}, align=center] (datlen) {Data\\Length};
\node [below, align=center] (bytes) at (datlen.south) {(Bytes)};

\path[name path=horiz7](datlen.west) -- ++(-5em,0);
\draw[->,thick,black,name intersections={of=Resvln and horiz7,by={Int7}}] (datlen.west) -- (Int7);

\path[name path=horiz8](datlen.east) -- ++(5em,0);
\draw[->,thick,black,name intersections={of=CtlDat and horiz8,by={Int8}}] (datlen.east) -- (Int8);


\path[name path=DatEndGuide] (canframe-1-27.north east) -- ++(0,8.7em);
\draw[thick,black,name path=DatEnd, name intersections={of=DatEndGuide and tophoriz,by={Top}}] (canframe-1-27.south east) -- (Top);

%Data
\node [above = 6em of {$(canframe-1-20.west)!0.5!(canframe-1-27.east)$}, align=center] (data) {Data};
\node [below, align=center] (bytes) at (data.south) {(8 Bytes Max)};

\path[name path=horiz9](data.west) -- ++(-5em,0);
\draw[->,thick,black,name intersections={of=CtlDat and horiz9,by={Int9}}] (data.west) -- (Int9);

\path[name path=horiz10](data.east) -- ++(5em,0);
\draw[->,thick,black,name intersections={of=DatEnd and horiz10,by={Int10}}] (data.east) -- (Int10);


\path[name path=DatEndGuide] (canframe-1-27.north east) -- ++(0,8.7em);
\draw[thick,black,name path=DatEnd, name intersections={of=DatEndGuide and tophoriz,by={Top}}] (canframe-1-27.south east) -- (Top);

\node [above = 5em of {$(canframe-1-28.west)!0.5!(canframe-1-31.east)$}, align=center] (ROF) {Rest of \\Frame};

\draw[->,thick,black] (canframe-1-27.east) |- ++(5em,4em);
\draw[thick,black] (canframe-1-1.north west) -- (canframe-1-31.north east);
\draw[thick,black] (canframe-1-1.south west) -- (canframe-1-31.south east);

\begin{scope}[yshift=-8em]
\matrix(dats)[matrix of nodes, nodes in empty cells, align=center, text width=0.7em]
{
		\tiny 1 & \tiny 2 & \tiny 3 & \tiny 4 & \tiny 5 & \tiny 6 & \tiny 7 & \tiny 8 & \tiny 9 & \tiny 10 & \tiny 11 & \tiny 12 & \tiny 13 & \tiny 14 & \tiny 15 & \tiny 16 & \tiny 17 & \tiny 18 & \tiny 19 & \tiny 20 & \tiny 21 & \tiny 22 & \tiny 23 & \tiny 24 & \tiny 25 & \tiny 26 & \tiny 27 & \tiny 28 & \tiny 29 & \tiny 30 & \tiny 31 & \tiny 32
		\\
	0& 1 & 1 & 0 & 1 &
	1 & 1 & 0 & 0 & 1 & 0 & 0 &	1 & 0 & 0 & 0 &
	0 & 0 & 1 & 1 & 1 & 1 & 1 & 1 &
	1 & 0 & 0 & 1 & 0 & 1 & 1 & 0 \\
};
\end{scope}
\draw[dashed, black] (canframe-1-20.south west) to[out=-120, in=90, distance = 5em] (dats-2-1.north west);
\draw[dashed, black] (canframe-1-23.south east) to[out=-60, in=90, distance = 5em] (dats-2-32.north east);
\draw[thick,black] (dats-2-1.north west) -- (dats-2-32.north east);
\draw[thick,black] (dats-2-1.south west) -- (dats-2-32.south east);

\draw[thick,black,name path=startType] (dats-2-1.north west) -- ++(0,-5em);
\draw[thick,black,name path=startTarget] (dats-2-6.north west) -- ++(0,-5em);
\draw[thick,black,name path=startID] (dats-2-17.north west) -- ++(0,-25em);
\draw[thick,black,name path=startFlags] (dats-2-25.north west) -- ++(0,-25em);
\draw[thick,black,name path=endFlags] (dats-2-32.north east) -- ++(0,-25em);

%Message Type
\node [below = 3em of {$(dats-2-1.west)!0.5!(dats-2-5.east)$}, align=center,style={text height=0.7em, text depth = 0em}] (MType) {Message\\ Type};

\path[name path=MTypeL](MType.west) -- ++(-1.5em,0);
\draw[->,thick,black,name intersections={of=startType and MTypeL,by={Int11}}] (MType.west) -- (Int11);

\path[name path=MTypeR](MType.east) -- ++(39em,0);
\draw[->,thick,black,name intersections={of=startTarget and MTypeR,by={Int12}}] (MType.east) -- (Int12);


%Target ID
\node [below = 3em of {$(dats-2-6.west)!0.5!(dats-2-16.east)$}, align=center,style={text height=0.7em, text depth = 0em}] (Target) {Target Node ID / Global};

\draw[->,thick,black] (Target.west) -- (Int12);

\draw[->,thick,black,name intersections={of=startID and MTypeR,by={Int13}}] (Target.east) -- (Int13);


%Error
\node[below right] (ERR) at (dats-2-17.south west){\footnotesize ERROR};

%Error ID
\node [below = 3em of {$(dats-2-17.west)!0.5!(dats-2-24.east)$}, align=center,style={text height=0.7em, text depth = 0em}] (Error) {Error ID};

\draw[->,thick,black] (Error.west) -- (Int13);

\draw[->,thick,black,name intersections={of=startFlags and MTypeR,by={Int14}}] (Error.east) -- (Int14);

%Error Hatch
\fill[color=black, opacity=0.1] (dats-2-17.south west) rectangle ($(Int14) +(0,-1em)$);
\end{tikzpicture}
lololololololololo
